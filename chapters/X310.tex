\twocolumn
\chapter{USRP X310 in MATLAB}
\ChapterCredits{
    Prabhat Kukunuri,
    Varun Shakunaveti,
    Anantha Krishnan
}

\section{Overview}

The Universal Software Radio Peripheral (USRP) X310, developed by Ettus Research,
is a reconfigurable software-defined radio (SDR) platform designed for real-time
signal processing and high-bandwidth wireless communication experimentation. Its
FPGA-based design, support for wideband radios, and compatibility with
open-source tools make it especially suitable for cognitive radio, MIMO systems,
and radar research.

By integrating flexible RF modules, the system can cover a wide frequency range
and adapt to diverse wireless standards. The X310 architecture enables
high-speed streaming through 10~Gigabit Ethernet, supporting low-latency
data transfer for demanding research applications.

\section{Hardware Architecture}

The core of the X310 is the Xilinx Kintex-7 FPGA, which performs computationally
intensive baseband operations such as filtering, modulation, and timing
synchronization. The system supports dual RF daughterboards with independent
transmit and receive chains.

Key hardware components include:
\begin{itemize}
    \item Xilinx Kintex-7 (XC7K410T) FPGA
    \item Dual DACs (16-bit) and dual ADCs (14-bit)
    \item 10~Gigabit Ethernet and PCIe~x4 host interfaces
    \item GPSDO / external reference timing support
\end{itemize}

\section{Software Stack and UHD Integration}

The USRP Hardware Driver (UHD) provides the primary software interface to the
USRP X310. UHD exposes a unified API for configuration, streaming, tuning, gain
control, and synchronization, and integrates with several higher-level tools:
\begin{itemize}
    \item GNU Radio
    \item MATLAB/Simulink
    \item C/C++ and Python-based applications
\end{itemize}

In the experiments described in this chapter, UHD is accessed primarily through
MATLAB support packages, which internally rely on UHD for device discovery,
streaming, and control.

\section{Host and Network Setup}

This section describes the practical steps required to power, network, and
probe the USRP X310 from a host machine.

\subsection*{Power and Basic Connectivity}

\begin{enumerate}
    \item Connect the power supply to the USRP X310 and then to the AC mains.
    \item Switch on the USRP using the front-panel power control (if available).
    Status LEDs indicate power and initialization state.
    \item On the host PC, configure the Ethernet interface to communicate with
    the USRP via UHD. Select the Ethernet interface connected to the USRP and
    change the IP assignment mode to \textbf{Manual} (or \textbf{Static}) with
    the following IPv4 settings:
    \begin{itemize}
        \item IPv4 Address: \texttt{192.168.10.1}
        \item Subnet Mask: \texttt{255.255.255.0}
        \item Default Gateway: (leave blank or set to \texttt{0.0.0.0})
    \end{itemize}
    After applying these settings, the host and the USRP reside on the same
    subnet. Basic connectivity can be verified by issuing a \texttt{ping}
    command to the USRP’s default IP address:
    \begin{lstlisting}
ping 192.168.10.2
    \end{lstlisting}
    A representative successful output is:
    {\footnotesize
    \begin{lstlisting}
Pinging 192.168.10.2 with 32 bytes of data:
Reply from 192.168.10.2: bytes=32 time=1ms TTL=32
Reply from 192.168.10.2: bytes=32 time=3ms TTL=32
Reply from 192.168.10.2: bytes=32 time=3ms TTL=32
Reply from 192.168.10.2: bytes=32 time=3ms TTL=32

Ping statistics for 192.168.10.2:
    Packets: Sent = 4, Received = 4, Lost = 0 (0% loss),
Approximate round trip times in milli-seconds:
    Minimum = 1ms, Maximum = 3ms, Average = 2ms
    \end{lstlisting}
    }
    The absence of packet loss confirms that the network configuration is
    correct and the device is reachable.
    \item Use the UHD probing utility to confirm that the device and
    daughterboards are correctly recognized:
    \begin{lstlisting}
uhd_usrp_probe --args="addr=192.168.10.2"
    \end{lstlisting}
    This command reports information such as device type, serial number,
    installed daughterboards, clock sources, and available channels.
\end{enumerate}

\subsection*{Firmware and FPGA Image Update}

For reliable operation, the USRP X310 should run a firmware and FPGA image
version compatible with the installed UHD release. If a version mismatch is
detected, UHD utilities can be used to update the images:

\begin{enumerate}
    \item Install the appropriate UHD version on the host PC.
    \item Download or locate the device images (firmware and FPGA bitstreams)
    distributed with UHD.
    \item Use the image loading utility to program the X310:
    {\footnotesize
    \begin{lstlisting}
uhd_image_loader --args="type=x300,addr=192.168.10.2"
    \end{lstlisting}
    }
    \item After the image is successfully written, power-cycle the USRP X310.
    \item Re-run \texttt{uhd\_usrp\_probe} to confirm that the device now
    reports the expected image versions.
\end{enumerate}

\section{MATLAB Integration}

This section focuses on the MATLAB software requirements and the specific radio
interfaces used in the experiments.

\subsection*{Required Toolboxes and Support Packages}

To interface USRP radios such as the X310 from MATLAB and conduct SDR
experiments, the following components are required:
\begin{itemize}
  \item MATLAB (base environment)
  \item \textbf{Communications Toolbox} --- for modulation/demodulation, filters,
        and baseband operations
  \item \textbf{Signal Processing Toolbox} --- for DSP operations, filtering,
        FFT, and spectral analysis
  \item \textbf{DSP System Toolbox} --- for advanced DSP functions, streaming,
        block-based systems, and real-time data handling
  \item \textbf{Wireless Testbench Support Package for NI USRP Radios} ---
        for X-series (X300 / X310 / X410) or N3xx-series USRPs
\end{itemize}

\subsection*{sdruTransmitter and sdruReceiver Interface}

The \texttt{sdruTransmitter} and \texttt{sdruReceiver} System objects in MATLAB
provide a legacy interface to USRP hardware for wireless baseband transmission
and reception. These objects allow streaming of IQ sample frames between MATLAB
and the USRP radio using the UHD driver.

\subsubsection*{Overview of Transmitter and Receiver Objects}

\begin{itemize}
    \item \textbf{\texttt{sdruTransmitter}}: used to send complex baseband
    samples from MATLAB to the USRP RF front-end. Key parameters include center
    frequency, gain, interpolation factor, and channel mapping.
    
    \item \textbf{\texttt{sdruReceiver}}: used to capture IQ samples from the
    USRP into MATLAB for processing. Important properties include center
    frequency, gain, decimation factor, sample rate, and frame size.
\end{itemize}

Once configured, both objects operate using the \texttt{step()} function inside
a streaming loop. The loop ensures continuous transmission or reception of
frames until manually stopped.

\subsubsection*{Example: Transmit and Receive IQ Frames}

The following simple example illustrates the creation of transmitter and
receiver objects, followed by a \texttt{for}-loop for real-time transmission and
reception of signals:

\begin{lstlisting}[language=MATLAB]
% Create transmitter and receiver objects
tx = sdruTransmitter('192.168.10.2', ...
    'CenterFrequency', 2.4e9, ...
    'Gain', 20);

rx = sdruReceiver('192.168.10.2', ...
    'CenterFrequency', 2.4e9, ...
    'Gain', 15, ...
    'SamplesPerFrame', 1024);

% Generate a baseband test signal
txSignal = exp(1j*2*pi*0.01*(0:1023)).';

% Streaming loop
for k = 1:100
    % Transmit a frame
    step(tx, txSignal);

    % Receive a frame
    rxSignal = step(rx);
    
    % Simple processing (e.g., display received energy)
    disp(mean(abs(rxSignal).^2));
end

% Release radio objects
release(tx);
release(rx);
\end{lstlisting}

The transmit object must be called continuously inside the streaming loop;
otherwise, the USRP will run out of samples and the transmission will stop due
to a buffer underflow condition.

In some cases, especially after power cycling the USRP or restarting MATLAB, it
is necessary to reload the FPGA image on the USRP X310 before streaming can
begin. This ensures that the firmware and image versions are synchronized with
the installed UHD and MATLAB support package. The following command can be used
to reconfigure the image on the device before running the transmit and
receive operations:

% {\scriptsize
\begin{lstlisting}
status=sdruload('Device','X310','IPAddress','192.168.10.2');
\end{lstlisting}
% }
A successful execution of this command confirms that the X310 is correctly
programmed and ready for baseband transmission and reception.

\subsection*{Wireless Testbench sdrtx / sdrrx Interface}

In addition to the legacy \texttt{sdru} interface, newer MATLAB workflows rely
on the Wireless Testbench \texttt{sdrtx} and \texttt{sdrrx} System objects. These
objects integrate directly with the support package for NI USRP radios and
provide convenient functions such as \texttt{transmitRepeat}.

Before executing scripts that use these objects, the USRP X310 must first be
initialized as part of the Wireless Testbench Support Package installation.
During this process, the Radio Setup wizard is launched from
\texttt{Add-Ons}~$\rightarrow$~\texttt{Manage Add-Ons}~$\rightarrow$~\texttt{Hardware Setup},
where the user selects the NI USRP hardware and configures the connection
settings. The wizard automatically detects the X310 on the network, verifies
IP connectivity, and updates the FPGA image and firmware if necessary. After
this initialization, MATLAB transmitter and receiver System objects can
communicate with the USRP using the assigned device identifier or IP address.

\section{Baseband Transmitter and Receiver Design}

In the proposed setup, the USRP X310 is controlled from MATLAB and operates as
a baseband transmitter--receiver pair. A complex baseband frame is first
generated in MATLAB and then uploaded to the radio. The transmitter is
configured to continuously retransmit this buffer, while the receiver runs in a
loop and acquires incoming IQ samples for further processing.

\subsection*{Baseband Transmitter}

The baseband transmitter generates a complex-valued IQ sequence, which may
represent a test tone, a modulated symbol stream, or a training sequence. This
frame is sent once to the USRP using a transmit object. Instead of calling the
transmitter inside a loop, the hardware is instructed to repeat the same frame
indefinitely, so the transmit operation runs in the background on the USRP.

\subsection*{Baseband Receiver}

The baseband receiver is implemented as a corresponding receive object in
MATLAB. The receiver is executed inside a \texttt{for}-loop, where a new frame
of IQ samples is captured from the USRP in each iteration. These received
frames can then be used for spectrum analysis, synchronization, correlation
with a known preamble, or bit-error rate (BER) computation. In this way, the
transmitter operates continuously, while the receiver processes a stream of
frames in real time.

\subsection*{Baseband Transmitter and Receiver Example in MATLAB}

% {\scriptsize
\begin{lstlisting}[language=MATLAB]
%% Baseband Transmitter and Receiver with USRP X310
% Assumes:
% - Wireless Testbench / USRP support package installed
% - USRP X310 reachable at the given device name or IP

fc       = 2.4e9;     % Center frequency
fs       = 1e6;       % Sample rate
txGain   = 20;        % Transmit gain (dB)
rxGain   = 15;        % Receive gain (dB)
frameLen = 2048;      % Samples per frame
numFrames = 200;      % Number of frames to receive

%% Create Transmitter Object
tx = sdrtx('USRP X310', ...
    'CenterFrequency', fc, ...
    'Gain',           txGain, ...
    'BasebandSampleRate', fs);

%% Create Receiver Object
rx = sdrrx('USRP X310', ...
    'CenterFrequency',     fc, ...
    'Gain',                rxGain, ...
    'BasebandSampleRate',  fs, ...
    'SamplesPerFrame',     frameLen, ...
    'OutputDataType',     'double');

%% Generate Baseband Transmit Frame (example: complex tone)
n       = (0:frameLen-1).';
txFrame = exp(1j*2*pi*0.01*n);   % simple complex exponential
txFrame = txFrame ./ max(abs(txFrame));  % normalize

%% Start Continuous Transmission on Hardware
transmitRepeat(tx, txFrame);
disp('Tx is continuously replaying the baseband frame...');

%% Continuous Reception Loop
for k = 1:numFrames
    rxSig = rx();
    
    % compute and display average received power
    rxPower = mean(abs(rxSig).^2);
    fprintf('Frame %d: Rx power = %.3f\n', k, rxPower);
end

%% Stop Transmission and Release Hardware
release(tx);
release(rx);
\end{lstlisting}

\section{Difficulties Faced}

During the implementation of the baseband transmitter and receiver using the
USRP X310 and MATLAB, several practical difficulties were encountered. These
issues affected the signal levels, workflow efficiency, and flexibility of the
software environment.

\subsection*{Low Transmit and Receive Power}

One of the main issues observed was that the received and transmitted signal
powers were significantly lower than expected. Even after successful
configuration of the center frequency and sample rate, the effective RF power
at the receiver side remained low. This was primarily attributed to
insufficient gain settings on the transmitter and receiver chains. As a result,
the received signal-to-noise ratio (SNR) was degraded, making it more difficult
to reliably observe and process the transmitted waveform without further
amplification or gain adjustment.

\subsection*{Frequent FPGA Image Reconfiguration}

Another notable difficulty was the frequent need to re-burn or reload the FPGA
image on the USRP X310 from within MATLAB. In order to switch between different
configurations or dependencies (e.g., different MATLAB support packages, UHD
versions, or experimental setups), the image often had to be reconfigured using
commands such as:
% {\footnotesize
\begin{lstlisting}
status = sdruload('Device','X310','IPAddress','192.168.10.2');
\end{lstlisting}
% }
This repeated image loading added considerable overhead to the workflow and
increased setup time before each experiment, especially when the device was
power-cycled or when changing between multiple test scenarios.

\subsection*{Scripting Software Limitations}

Another major difficulty encountered was related to the MATLAB scripting
environment itself. Since MATLAB does not natively support multithreading for
System object execution, the entire transmit--receive chain runs as a single
threaded process. This significantly limits real-time performance when handling
high-rate streaming or processing-intensive algorithms. As a result, the CPU
workload frequently became a bottleneck, restricting the maximum achievable
sampling rates and the responsiveness of the system.

Additionally, the availability of detailed documentation and example resources
specifically for the USRP X310 in MATLAB was limited. Most online support and
open-source code repositories were focused on GNU Radio or UHD C++ workflows,
making it difficult to troubleshoot MATLAB-specific integration issues. This
lack of reference material and debugging guidance contributed to increased
development time and experimentation overhead.
