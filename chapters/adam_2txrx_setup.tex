\section{Python Driver Installation and Configuration}

This section describes the installation and configuration of PlutoSDR drivers for Python across Windows, Linux, and macOS platforms. The recommended approach uses Miniconda-based virtual environments to ensure reproducibility and avoid dependency conflicts.

\subsection{Prerequisites}

Before proceeding, ensure familiarity with basic Python programming, digital signal processing concepts, and command-line usage. The PlutoSDR must be physically connected via USB and recognized by the operating system.

\subsection{Cross-Platform Installation Guidance}

General installation guidance applicable to all platforms is maintained by Analog Devices:

\begin{lstlisting}
https://wiki.analog.com/sdrseminars
\end{lstlisting}

This resource contains platform-specific driver requirements, troubleshooting steps, and supported software versions.

\subsection{Recommended Installation for Windows}

The following procedure provides a reliable installation path for Windows users using Miniconda for environment management.

\subsubsection{Install Miniconda}

Download and install Miniconda from the official Anaconda distribution:

\begin{lstlisting}
https://www.anaconda.com/docs/getting-started/miniconda/install
\end{lstlisting}

Miniconda provides a minimal Python distribution with the \texttt{conda} package manager, suitable for isolated environments.

\subsubsection{Create and Activate a Virtual Environment}

Open Command Prompt or PowerShell and execute:

\begin{lstlisting}[language=bash]
conda create -n pluto python=3.10
conda activate pluto
\end{lstlisting}

Python 3.10 is recommended for compatibility with current \texttt{pyadi-iio} releases.

\subsubsection{Install PyADI-IIO}

With the environment active, install the Analog Devices Python interface:

\begin{lstlisting}[language=bash]
pip install pyadi-iio
\end{lstlisting}

This installs the \texttt{adi} module used to interface with the PlutoSDR.

\subsubsection{Verify Installation}

Create a test script \texttt{pluto\_test.py}:

\begin{lstlisting}[language=Python]
import numpy as np
import adi

sample_rate = 1e6
center_freq = 915e6

sdr = adi.Pluto("ip:192.168.2.1")
sdr.sample_rate = int(sample_rate)
sdr.tx_rf_bandwidth = int(sample_rate)
sdr.tx_lo = int(center_freq)
sdr.tx_hardwaregain_chan0 = -50

N = 10000
t = np.arange(N) / sample_rate
samples = 0.5 * np.exp(2j * np.pi * 100e3 * t)
samples *= 2**14

for _ in range(100):
    sdr.tx(samples)
\end{lstlisting}

Run:

\begin{lstlisting}[language=bash]
python pluto_test.py
\end{lstlisting}

Successful execution indicates correct installation and connectivity.

\begin{DndComment}{Common Issues}
If you encounter \texttt{ModuleNotFoundError: No module named 'adi'}, verify that the Conda environment is active and \texttt{pyadi-iio} is installed within it.
\end{DndComment}

\subsubsection{Extended TX/RX Test}

Example TX/RX scripts are available at:

\begin{lstlisting}
https://pysdr.org/content/pluto.html
\end{lstlisting}

RX and TX hardware gains may require manual tuning for clean visualization.

\section{Frequency Range Extension}

By default, the PlutoSDR supports approximately 325~MHz to 3.8~GHz. The underlying RFIC is capable of operation from 70~MHz to 6~GHz, which can be unlocked through firmware configuration.

\subsection{Performance Considerations}

Operation outside the official frequency range is not guaranteed. Performance degradation, increased phase noise, and instability may occur, particularly above 5~GHz.

\subsection{Configuration Procedure}

\subsubsection{Verify Firmware Version}

Firmware version 0.31 or newer is required:

\begin{lstlisting}
https://wiki.analog.com/university/tools/pluto/users/firmware
\end{lstlisting}

\subsubsection{Enable Extended Tuning}

SSH into the device:

\begin{lstlisting}[language=bash]
ssh root@192.168.2.1
\end{lstlisting}

Run:

\begin{lstlisting}[language=bash]
fw_setenv attr_name compatible
fw_setenv attr_val ad9364
reboot
\end{lstlisting}

After reboot, verify tuning outside the default range.

\section{Dual Transmit and Receive Configuration (2TX/2RX)}

The AD9361 supports two TX and two RX chains internally, though only one of each is routed externally on stock hardware.

\subsection{Hardware Modification Requirements}

Full 2TX/2RX operation requires hardware modification:

\begin{lstlisting}
https://wiki.analog.com/university/tools/pluto/hacking/hardware
\end{lstlisting}

\begin{DndReadAloud}
Hardware modification voids warranty and carries risk. Proceed only if you have appropriate soldering experience.
\end{DndReadAloud}

\subsection{Firmware Configuration}

Enable 2TX/2RX mode:

\begin{lstlisting}[language=bash]
ssh root@192.168.2.1
fw_setenv attr_name compatible
fw_setenv attr_val ad9361
fw_setenv mode 2r2t
reboot
\end{lstlisting}

\subsection{Python Configuration Example}

\begin{lstlisting}[language=Python]
import numpy as np
import adi

sdr = adi.ad9361("ip:192.168.2.1")
sdr.sample_rate = int(2e6)

sdr.tx_lo = int(2.4e9)
sdr.tx_enabled_channels = [0, 1]
sdr.tx_hardwaregain_chan0 = -30
sdr.tx_hardwaregain_chan1 = -30

sdr.rx_lo = int(2.4e9)
sdr.rx_enabled_channels = [0, 1]
sdr.rx_hardwaregain_chan0 = 20
sdr.rx_hardwaregain_chan1 = 20

N = 1024
t = np.arange(N) / sdr.sample_rate
tx0 = np.exp(2j * np.pi * 100e3 * t) * 2**14
tx1 = np.exp(2j * np.pi * 200e3 * t) * 2**14

sdr.tx([tx0, tx1])
rx_data = sdr.rx()
\end{lstlisting}

\subsection{MATLAB Configuration Summary}

MATLAB requires the Analog Devices Transceiver Toolbox and the AD9361 system object:

\begin{lstlisting}[language=Matlab]
tx = adi.AD9361.Tx('uri','ip:192.168.2.1');
tx.EnabledChannels = [1,2];

rx = adi.AD9361.Rx('uri','ip:192.168.2.1');
rx.EnabledChannels = [1,2];
\end{lstlisting}

\subsection{Reference Implementations}

\begin{DndTable}[header=Selected Resources]{lX}
\textbf{Topic} & \textbf{Resource} \\
2TX/2RX Setup & \href{https://www.youtube.com/watch?v=ph0Kv4SgSuI}{YouTube} \\
Phased Arrays & \href{https://github.com/jonkraft}{Jon Kraft GitHub} \\
ADI Python Examples & \href{https://github.com/analogdevicesinc/pyadi-iio/tree/master/examples}{GitHub} \\
\end{DndTable}
