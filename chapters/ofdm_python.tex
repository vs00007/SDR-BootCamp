\chapter{Python Examples}
\section{OFDM implementation}

This section discusses some important code blocks and functions used in the implementation of OFDM in pySDR on a single Adalm Pluto device.


% ============================================================================
\section{Pilot Symbol Generation}

\DndSpellHeader%
  {pilot\_set()}
  {Pilot Initialization}
  {OFDM\_mask, power\_scaling}
  {Transmitter}
  {PyTorch}
  {QPSK Pilots Only}

\begin{lstlisting}
def pilot_set(OFDM_mask, power_scaling=1.0): ###S
# Define QPSK pilot values
pilot_values = torch.tensor([-0.7 - 0.7j, -0.7 + 0.7j, 0.7 - 0.7j, 0.7 + 0.7j]) * power_scaling

# Count the number of pilot elements in the OFDM_block mask
num_pilots = OFDM_mask[OFDM_mask == 2].numel()

# Create and return a list of pilot values repeated to match the number of pilots
return pilot_values.repeat(num_pilots // 4 + 1)[:num_pilots]
\end{lstlisting}

This code sets the pilots for QPSK modulation. In this OFDM implementation, every
subcarrier is modulated by QPSK modulation. These pilots will later serve the
purpose of timing synchronization and channel estimation.

% ============================================================================
\section{OFDM Frame Generation}

\DndSpellHeader%
  {create\_OFDM\_data()}
  {OFDM Frame Generator}
  {None}
  {Transmitter}
  {IFFT, RE mapping, CP addition}
  {Fixed FFT size and CP}

\begin{lstlisting}
def create_OFDM_data():
pdsch_bits, pdsch_symbols = create_payload(OFDM_mask, Qm, mapping_table_Qm, power=1) # create PDSCH data and modulate it
Modulated_TTI = RE_mapping(OFDM_mask, pilot_symbols, pdsch_symbols, plotOFDM_block=True) # map the PDSCH and pilot symbols to the TTI
TD_TTI_IQ = IFFT(Modulated_TTI) # perform the FFT
TX_Samples = CP_addition(TD_TTI_IQ, S, FFT_size, CP) # add the CP
if use_sdr:
    zeros = torch.zeros(leading_zeros, dtype=TX_Samples.dtype) # create leading zeros for estimating noise floor power
    TX_Samples = torch.cat((zeros, TX_Samples), dim=0) # add leading zeros to TX samples
return pdsch_bits, TX_Samples

pdsch_bits, TX_Samples = create_OFDM_data()
\end{lstlisting}

This code block generates the symbol frames for ofdm transmission. The bits are
mapped to complex symbols by QPSK modulation through mapping table, where the
function RE mapping, is used to allocate pilot and data symbols. After completing
the mapping, the next step is to add cyclic prefix.

% ============================================================================
\section{Transmission Using Pluto SDR}

After initialization and setting up the Pluto SDR, the next step is to transmit
the ofdm frame.

\DndSpellHeader%
  {radio\_channel()}
  {RF Transmission Interface}
  {tx\_signal, tx\_gain, rx\_gain, ch\_SINR}
  {Transmitter--Receiver}
  {Pluto SDR / Channel Model}
  {Cyclic Transmission Enabled}

\begin{lstlisting}
def radio_channel(use_sdr, tx_signal, tx_gain, rx_gain, ch_SINR):
if use_sdr:
    if randomize_tx_gain:
        tx_gain = random.randint(tx_gain_lo, tx_gain_hi) 
        # add random gain to the TX gain
    SDR_1.SDR_gain_set(tx_gain, rx_gain)
    # set the gains
    print("TX Gain: ", tx_gain, ", RX Gain: ", rx_gain)
    SDR_1.SDR_TX_send(SAMPLES=tx_signal, max_scale=TX_Scale, cyclic=True) 
    # start transmitting the symbols in cyclic mode
    time.sleep(0.2) 
    # Transmit for 500ms longer
    rx_signal = SDR_1.SDR_RX_receive(len(tx_signal)*4) 
    # receive the signal with 4 times more samples than the length of the transmitted signal.
    SDR_1.SDR_TX_stop() 
    # stop the transmission
else:
    rx_signal, h = 
        apply_multipath_channel_dop(tx_signal, 
           max_n_taps=n_taps, 
           max_delay=max_delay_spread, 
           random_start=True, 
           repeats=3, 
           SINR=ch_SINR, 
           leading_zeros=leading_zeros,
           fc=SDR_TX_Frequency, 
           velocity=velocity,
           fs=SampleRate,
           randomize=False)
    print(h)
return rx_signal

RX_Samples = radio_channel(use_sdr=use_sdr, tx_signal = TX_Samples, tx_gain = tx_gain, rx_gain = rx_gain, ch_SINR=ch_SINR)
\end{lstlisting}

This code block transmits the desired symbols through pluto and recieves the
adequate amount of symbols. If cyclic = True, the SDR transmits indefinitely
until the buffer is destroyed, we would want that condition, since we want to
receive for multiple cycles and get the best possible result. It is worth noting
to add a sleep time(within 1 sec) to let the SDR carry on with it's buffer
synchronization and some hardware works, not adding this sleep would cause the
code to fail and give incorrect results.

% ============================================================================
\section{Frame Synchronization}

\DndSpellHeader%
  {sync\_iq()}
  {Frame Synchronization}
  {tx\_signal, rx\_signal, leading\_zeros}
  {Receiver}
  {Cross-correlation}
  {Known Transmitted Frame}

\begin{lstlisting}
def sync_iq(tx_signal, rx_signal, leading_zeros, threshold_factor=6):
tx_len = tx_signal.numel()
rx_len = rx_signal.numel()
end_point = rx_len - tx_len
rx_signal = rx_signal[leading_zeros:end_point]

corr_result_real = tFunc.conv1d(rx_signal.real.view(1, 1, -1), tx_signal.real.view(1, 1, -1)).view(-1)
corr_result_imag = tFunc.conv1d(rx_signal.imag.view(1, 1, -1), tx_signal.imag.view(1, 1, -1)).view(-1)
correlation = torch.complex(corr_result_real, corr_result_imag).abs()

threshold = correlation.mean() * threshold_factor

i_maxarg = torch.argmax(correlation).item() + leading_zeros

# Find the first index where correlation exceeds threshold
for i, value in enumerate(correlation):
    if value > threshold:
        sync_index = i
        break
else:
    sync_index = 0

return sync_index + leading_zeros, i_maxarg, correlation, threshold
\end{lstlisting}

This function is used for frame synchronization. This method can only be used
when you already know the transmitted frame, since there is no barker/preamble
sequence used. The code finds cross correlation between iq samples of received and
transmitted frames to find where the reception is starting. We have to set a
particular threshold above which we know that the pilot signals have been
detected. i.e this code block particularly points us to where the first preamble
detection happens. So, after this detection, we would know where our symbols (i.e
data symbols, pilot symbols within the blocks and cyclic prefixes) start. It is
worth noting that, this is different from timing synchronization, this tells us
where sampling indices inside an ofdm symbol/single carrier.

% ============================================================================
\subsection{Cyclic Prefix Removal}

The next step is cyclic prefix (CP) removal. Once frame synchronization is
completed and the start of each OFDM symbol is known, this step becomes
straightforward. For every received OFDM symbol, we simply discard (or mask out)
the first $N_{\text{CP}}$ samples corresponding to the cyclic prefix and retain
the following $N_{\text{FFT}}$ samples, which contain the useful data portion of
the symbol.

% ============================================================================
\subsection{Channel Estimation and equalization}

Since the wireless channel inherently introduces several distortions to the
transmitted signal, such as attenuation, phase shift, multi path fading, and
frequency-selective distortion, it is necessary to estimate and compensate for
these anomalies at the receiver. This process is one of the most critical steps
in a digital communication system.

\begin{equation}
Y[k] = H[k]X[k] + W[k],
\end{equation}

Channel estimation refers to the process of estimating the unknown channel
response \(H[k]\). At pilot subcarriers, where the transmitted symbols
\(X_p[k]\) are known at the receiver, the channel is estimated using the least
squares (LS) estimator as

\begin{equation}
\hat{H}_p[k] = \frac{Y_p[k]}{X_p[k]}.
\end{equation}

This estimation cannot be performed on all subcarriers since the transmitted data
symbols are unknown at the receiver. Therefore, pilot symbols are inserted at
certain known subcarriers. The channel is first estimated at these pilot
locations and then interpolated across the frequency axis to obtain the channel
estimate over the entire frame.

Once the channel has been estimated, equalization is performed to recover the
transmitted data symbols using

\begin{equation}
\hat{X}[k] = \frac{Y[k]}{\hat{H}[k]}.
\end{equation}

In this implementation, linear interpolation is used to estimate the channel
values between two pilot subcarriers. The interpolation is given by

\begin{equation}
\boxed{
\hat{H}(k) = \hat{H}(k_1) +
\frac{\hat{H}(k_2) - \hat{H}(k_1)}{k_2 - k_1}
(k - k_1)
}
\end{equation}

where \(k_1\) and \(k_2\) denote two adjacent pilot subcarrier indices.

Once this step is done, we need to remove the guard bands from the recievd
signal, later we perform zero forcing which is the equalization step. After
estimating the channel frequency response, the effect of the wireless channel
must be removed from the received signal in order to recover the transmitted data
symbols. This process is known as \textit{equalization}. In this work,
Zero-Forcing (ZF) equalization is employed.

\begin{equation}
Y[k] = H[k]X[k] + W[k],
\end{equation}

The Zero-Forcing equalizer compensates for the channel by directly inverting the
estimated channel response. The transmitted symbol is recovered as

\begin{equation}
\boxed{
\hat{X}[k] = \frac{Y[k]}{\hat{H}[k]}
}
\end{equation}

The estimated channel can be represented in polar form as

\begin{equation}
\hat{H}[k] = |\hat{H}[k]| e^{j\angle \hat{H}[k]}.
\end{equation}

Using this representation, the ZF equalization can also be expressed as

\begin{equation}
\boxed{
\hat{X}[k] = \frac{Y[k]}{|\hat{H}[k]|} e^{-j\angle \hat{H}[k]}
}
\end{equation}

Since the channel phase may contain discontinuities due to phase wrapping in
the range \([-\pi, \pi]\), phase unwrapping is applied prior to phase compensation
to ensure smooth phase correction across the subcarriers.

\begin{equation}
\hat{X}[n,k] = \frac{Y[n,k]}{\hat{H}[k]}, \quad n = 0,1,\dots,S-1,\;
k = 0,1,\dots,F-1,
\end{equation}

The output of the ZF equalizer produces constellation points that are ideally
clustered around their original QAM symbol locations, thereby enabling accurate
symbol detection and demodulation.

% ============================================================================
\subsection{Demapping step and BER calculation}

Once the channel estimation, guard band removal, equalization is completed, we
can now extract the data symbols from the frame using a demapping function. Which
maps the complex symbols into data. And this is the final step of the OFDM
communication pipeline.
\subsection{Important Notes}
\begin{itemize}
    \item This code only works for the offline setup, i.e requires the knowledge of transmitted symbols, since we use these to complete frame synchronization. 
    \item In any case, when ever running transmitter and reciever together, it is suggested to introduce a sleep time (in order of 1 second) so that you let the hardware synchronize.
    \item Since the original code contained some errors, we tried to change as much as possible and suggest the readers to change the errors try to implement an online version of the code though this involves making significant changes to the source code, we tried and failed to do so.
    Some of the things we tried are as follows:
    \item For the real-time implementation, the initial idea was to remove buffer-destruction steps and enable continuous transmission with live plotting. This was later replaced by a buffer-wise processing approach that applied the same offline pipeline sequentially to each buffer. PyQt was initially considered for real-time visualization but was dropped due to its complexity, and Matplotlib was used instead. Since Matplotlib is blocking by default, interactive mode using plt.ion() was required for non-blocking visualization.
    \item Once real-time spectral visualization was achieved, synchronization on a per-buffer basis became the next challenge. A workaround involving zero-padding before each buffer was tested, wherein correlation was computed only on the valid samples. While this worked for the first buffer with an SINR of 15–20 dB, all subsequent buffers failed, with the SINR dropping to zero, demonstrating the impracticality of this approach.
\end{itemize}