
\chapter{Cross-Platform Comms}
\ChapterCredits{
    Dhanush V Nayak
    Kaustubh Khachane
}
\section{The Big Picture}
We are building a security system where Alice (USRP B210) sends a secret message using two antennas, and Bob (Adalm Pluto) tries to receive it using one antenna.

\begin{conceptbox}{The Main Challenge}
The USRP and the Pluto are like two people who speak different dialects and hence different way of handling things. Just to list :
\begin{itemize}
    \item \textbf{USRP}Our Alice has to send and receive data and also do channel processing.
    \item \textbf{Pluto} Bob has to just receive and decode the data.
\end{itemize}
To make them understand each other and make good use of the hardware, we must force them to speak at exactly the same speed (3 MHz).
\end{conceptbox}

\section{Step 1: The Hardware Setup}

We use "Objects" in MATLAB to control the hardware. Here is how we configured them in \texttt{TX\_dual\_pluto.m} and \texttt{Hardware\_RX\_Bob.m}.

\vspace{0.5cm}

\begin{alicebox}{Alice: The Transmitter}
Alice is the "Master" of the transmission. She sends two signals at once.

\begin{itemize}
    \item \textbf{Device:} USRP B210
    \item \textbf{Goal:} Send separate streams on Antenna 1 and Antenna 2.
    \item \textbf{The Trick:} We use \textbf{Interpolation} to slow down the internal clock to a speed the Pluto can handle.
\end{itemize}

\textbf{Key Settings:}
\begin{enumerate}
    \item \texttt{MasterClockRate = 30 MHz} (Internal Heartbeat)
    \item \texttt{InterpolationFactor = 10} (Slow down by 10x)
    \item \textbf{Resulting Speed:} $30 \text{ MHz} / 10 = \mathbf{3 \text{ MHz}}$
    \item \texttt{ChannelMapping = [1 2]} (Use both antennas)
\end{enumerate}
\end{alicebox}

\vspace{0.5cm}

\begin{bobbox}{Bob: The Receiver}
Bob is the listener. He uses a cheaper radio, so we have to tell him exactly what to listen for.

\begin{itemize}
    \item \textbf{Device:} Adalm Pluto (PlutoSDR)
    \item \textbf{Goal:} Listen to Alice and separate her two signals.
    \item \textbf{The Trick:} We force the Pluto's "Baseband Rate" to match Alice exactly.
\end{itemize}

\textbf{Key Settings:}
\begin{enumerate}
    \item \texttt{BasebandSampleRate = 3 MHz} (Must match Alice!)
    \item \texttt{CenterFrequency = 800 MHz} (Must match Alice)
    \item \texttt{Gain = 60 dB} (High gain because Pluto is less sensitive)
\end{enumerate}
\end{bobbox}


\section{Step 2: How Bob Finds the Message}

The air is full of noise. Bob needs a way to know "Hey, a message is starting now!"

\subsection*{The Packet Detection}
Alice sends a special pattern called a Preamble at the start of every message. This acts like a message to capture the data.

\begin{conceptbox}{Schmidl-Cox Algorithm (Simple Explanation)}
\begin{enumerate}
    \item Alice sends a short pattern: \texttt{[A, A, A, A]}.
    \item Bob constantly compares the signal he just received with the signal he received a split second ago.
    \item If \textbf{Current Signal} looks exactly like \textbf{Past Signal}, Bob knows it's the repeating preamble.
    \item \textbf{Math:} We calculate a score $M(n)$. When $M(n) > 0.6$, we trigger the recording.
\end{enumerate}
\end{conceptbox}

\section{Step 3: Separating the Two Antennas}

This is the most critical part of Physical Layer Security. Bob has only \textbf{one ear} (antenna), but Alice is speaking with \textbf{two mouths} (Tx1 and Tx2). How does he know who said what?

\subsection*{The "Pilot" Solution}
We use \textbf{Orthogonal Pilots}.Please refer to the codes in the same folder for more detail.

\begin{center}
\begin{tabular}{|c|c|c|}
\hline
\rowcolor{gray!20} \textbf{Frequency Key} & \textbf{Tx1 Action} & \textbf{Tx2 Action} \\
\hline
Key \#16 & \textbf{Signal1} & Silent(Null) \\
\hline
Key \#18 & Silent & \textbf{Signal2} \\
\hline
Key \#24 & \textbf{Signal1} & Silent \\
\hline
Key \#26 & Silent & \textbf{Signal2} \\
\hline
\end{tabular}
\end{center}

\begin{enumerate}
    \item Bob listens to Key \#16. He hears a sound. Since he knows Tx2 is silent there, \textbf{that sound must describe Tx1's channel.}
    \item Bob listens to Key \#18. He hears a sound. Since he knows Tx1 is silent there, \textbf{that sound must describe Tx2's channel.}
\end{enumerate}

\section{Summary: The Code Logic}

Here is the flow of \texttt{OFDM\_RX\_Bob.m} in plain English:

\begin{tcolorbox}[colback=yellow!10, colframe=orange!80!black, title=The Receiver Logic Chain]
\begin{enumerate}
    \item \textbf{Wait} for the trigger signal ($M_n > 0.6$).
    \item \textbf{Cut} the signal into a "Frame" (480 samples).
    \item \textbf{Correct} the frequency (Pluto and USRP clocks are slightly different, so we rotate the signal back).
    \item \textbf{Extract} the Pilot Keys (indices 16, 18, 24, etc.).
    \item \textbf{Calculate} Channel 1 from the Tx1 keys.
    \item \textbf{Calculate} Channel 2 from the Tx2 keys.
    \item \textbf{Decode} the secret message using these channel estimates.
\end{enumerate}
\end{tcolorbox}