\chapter{Exercises for Digital Comms}

\ChapterCredits{
    Mihir Divyansh,
    Vemula Siddhartha
}

\section{Programming Exercises}


\subsection{Foundational Implementations}

\begin{enumerate}
\item \textbf{Frequency Offset Estimation and Correction}

Implement a frequency offset estimator and compensator for a complex baseband signal.

\begin{itemize}
\item Inject a known carrier frequency offset into a transmitted signal.
\item Estimate the offset using at least one method (FFT / Autocorr based)
\item Apply frequency correction and verify that constellation rotation is removed.
\item Measure the maximum CFO (as a fraction of symbol rate) for which your estimator remains accurate.
\end{itemize}

\item \textbf{Symbol Timing Detection}

Implement a basic symbol timing recovery mechanism without using built-in synchronization blocks.

\begin{itemize}
\item Evaluate at least one of the following approaches:
  \begin{itemize}
    \item Early--late gate
    \item Derivative or zero-crossing-based timing
  \end{itemize}
\item Visualize the eye diagram before and after timing correction.
\item Quantify timing error variance before and after convergence.
\end{itemize}

\item \textbf{Frame Detection via Correlation}

Implement a frame detector using correlation with a known preamble.

\begin{itemize}
\item Plot the correlation magnitude over time.
\item Identify false alarm and miss detection cases.
\item Measure detection delay relative to the first transmitted data symbol.
\end{itemize}

\item \textbf{Noise and SNR Estimation}

Develop an estimator for noise variance and SNR from received samples.

\begin{itemize}
\item Compare estimation accuracy for:
  \begin{itemize}
    \item Known pilot symbols
    \item Blind estimation using decision errors
  \end{itemize}
\item Quantify estimation bias at low SNR.
\end{itemize}

\end{enumerate}

\subsection{PSK}

\begin{enumerate}
\item \textbf{Carrier Recovery Under Dynamic Conditions}

Implement a QPSK receiver in which the carrier frequency offset (CFO) changes linearly over time (simulating oscillator drift).

\begin{itemize}
\item Design a tracking loop capable of following time$-$varying CFO rather than assuming a constant offset.
\item Compare the BER performance of this adaptive approach against a one$-$time coarse CFO correction.
\item At what rate of CFO change (Hz/s) does your tracking loop fail to converge?
\end{itemize}

\item \textbf{Frame Synchronization Trade$-$offs}

The text mentions Barker sequences for frame detection. Implement and compare the following preamble strategies:

\begin{itemize}
\item A Barker$-$11 sequence
\item A Zadoff$-$$-$Chu sequence of length 13
\item A repeated m$-$sequence of length 7 (transmitted twice)
\end{itemize}

For each preamble, measure:
\begin{itemize}
\item Probability of false detection under AWGN
\item Timing accuracy of the detected correlation peak
\item Performance degradation in the presence of uncorrected CFO
\end{itemize}

Which preamble would you choose for:
\begin{itemize}
\item A low$-$SNR channel?
\item A high$-$mobility (high Doppler) scenario?
\end{itemize}
\end{enumerate}

\subsection{OFDM }

\begin{enumerate}
\setcounter{enumi}{2}

\item \textbf{Pilot Pattern Design and Channel Estimation}

Implement three pilot insertion strategies in an OFDM system:

\begin{itemize}
\item Block$-$type: all subcarriers in selected OFDM symbols are pilots
\item Comb$-$type: every $N$th subcarrier in every symbol is a pilot
\item Scattered: pilots distributed across both time and frequency
\end{itemize}

Develop a channel estimation algorithm for each scheme. Test over a frequency$-$selective Rayleigh fading channel.

\begin{itemize}
\item Which pilot pattern yields the lowest BER?
\item Which is most robust to Doppler spread?
\end{itemize}

\item \textbf{PAPR Reduction Techniques}

OFDM exhibits high peak$-$to$-$average power ratio (PAPR), which may cause nonlinear distortion in the PlutoSDR transmit chain.

\begin{itemize}
\item Implement clipping$-$and$-$filtering for PAPR reduction.
\item Implement selective mapping (SLM): generate multiple phase$-$rotated OFDM symbols and transmit the one with the lowest PAPR.
\item Measure PAPR reduction and quantify out$-$of$-$band spectral regrowth.
\item Transmit both high$-$PAPR and PAPR$-$reduced OFDM signals at maximum TX gain. Can you observe nonlinear distortion in a loopback experiment?
\end{itemize}

\item \textbf{Subcarrier Nulling and Spectral Shaping}

Implement an OFDM transmission function that supports arbitrary spectral masks:

\begin{lstlisting}[language=Matlab]
function txFrame = ofdm_shaped_transmit(data, maskVector)
% maskVector: length N_subcarriers, values 0 (null) or 1 (active)
\end{lstlisting}

\begin{itemize}
\item Design a mask that nulls the center 20\% of subcarriers (creating a notch).
\item Implement power loading based on estimated channel gains. (for a static channel)
\item Verify that the transmitted spectrum respects the mask using \textbf{loopback} measurements.
\item Does subcarrier nulling affect PAPR and BER?
\end{itemize}

\item \textbf{Synchronization Failure Analysis}

Intentionally break the OFDM synchronization chain:

\begin{itemize}
\item Remove the cyclic prefix (CP) at the receiver only.
\item Introduce a timing offset of exactly $N/2$ samples (where $N$ is the FFT size).
\item Introduce a carrier offset equal to one subcarrier spacing.
\end{itemize}

For each case:
\begin{itemize}
\item Plot the received constellation.
\item Explain \emph{why} it appears as observed.
\item Identify which failure mode is easiest to detect automatically.
\end{itemize}

Design a diagnostic function that determines which synchronization stage has failed based only on corrupted received symbols.

\end{enumerate}

\subsection{Theory and Analysis Questions}

\begin{enumerate}
\setcounter{enumi}{6}

\item \textbf{Gray Coding and Bit Error Propagation}

The text claims that Gray coding ``ensures robustness,'' which is imprecise.

\begin{itemize}
\item Does Gray coding reduce symbol error rate, or only bit error rate for a given symbol error?
\item For 16$-$PSK, compare Gray coding and natural binary coding at high SNR. Calculate the average number of bit errors per symbol error.
\item Construct a scenario in which Gray coding increases BER (hint: non$-$AWGN channels or iterative decoding).
\end{itemize}

\item \textbf{RRC Filter Cascade and ISI}

Transmit and receive RRC filters combine to form a Raised Cosine response.

\begin{itemize}
\item If the transmitter uses an RRC filter with roll$-$off $\alpha=0.5$ and the receiver mistakenly uses $\alpha=0.35$, derive or simulate the combined impulse response. What happens to ISI?
\item Given the PlutoSDR's finite TX bandwidth, at what symbol rate does RRC spectral clipping begin? How does this affect the eye diagram?
\item Propose a loopback$-$based experimental method to verify correct RRC filter implementation.
\end{itemize}

\item \textbf{OFDM Orthogonality Under Hardware Impairments}

Real OFDM systems experience:

\begin{itemize}
\item Phase noise
\item I/Q imbalance
\item Timing jitter
\item Power amplifier nonlinearity
\end{itemize}

For each impairment, determine whether it primarily causes:

\begin{itemize}
\item Common phase error (CPE)
\item Inter$-$carrier interference (ICI)
\item In$-$band noise floor elevation
\item Constellation asymmetry or warping
\end{itemize}

Design an experiment using the PlutoSDR to measure and isolate these effects.

\item \textbf{Cyclic Prefix Length Selection}

\begin{itemize}
\item What is the impact of excessively long CP on spectral efficiency and SNR?
\item If the CP is too short by exactly three samples, derive the resulting residual ISI.
\item Estimate a reasonable CP length for an indoor environment at 2.4~GHz. How does this change for an outdoor urban scenario at 900~MHz?
\item Can variable USB latency affect the required CP length? Why or why not?
\end{itemize}

\end{enumerate}

\begin{DndReadAloud}
If thee can implementeth it but can't explain why it hath broken, thee has't only copied, not understood.
\end{DndReadAloud}
