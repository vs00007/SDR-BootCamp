%\chapter*{Appendix A: \\Pluto SDR Hardware Specifications}
%%This is prabhat just made a change so that the heading is visible in the contents the original name is commented above if need to revert.
\appendix
\chapter{Pluto SDR Hardware Specifications}
\DndDropCapLine{T}{he ADALM-PLUTO (PlutoSDR)} is an active learning module designed by Analog Devices for hands-on exploration of software-defined radio, RF, and wireless communications. This appendix provides comprehensive hardware specifications and configuration details essential for working with the platform.

\section{Core Hardware Specifications}

\subsection{RF Transceiver: AD9361}

At the heart of the PlutoSDR is the AD9361 integrated RF Agile Transceiver, providing full-duplex operation with exceptional flexibility.

\begin{DndTable}[header=AD9361 Key Specifications]{lX}
  \textbf{Parameter} & \textbf{Specification} \\
  Frequency Range & 325 MHz to 3.8 GHz (default) \\
  Extended Range & 70 MHz to 6.0 GHz (unofficial) \\
  Channel Bandwidth & 200 kHz to 56 MHz \\
  Sample Rate & 2.083333 MSPS to 61.44 MSPS \\
  ADC Resolution & 12-bit \\
  DAC Resolution & 12-bit \\
  Receive Channels & 2 (1 active by default) \\
  Transmit Channels & 2 (1 active by default) \\
\end{DndTable}

\begin{DndComment}{Frequency Range Note}
  The official specification limits operation to 325 MHz - 3.8 GHz. However, the AD9361 chip itself supports 70 MHz - 6 GHz. Extended range operation can be enabled through firmware modifications but is not guaranteed by Analog Devices and may have reduced performance outside the specified range.
\end{DndComment}

\subsection{Processing and Memory}

\begin{DndTable}[header=System Components]{lX}
  \textbf{Component} & \textbf{Specification} \\
  Main Processor & Xilinx Zynq Z-7010 SoC \\
  ARM Core & Dual-core ARM Cortex-A9 @ 800 MHz \\
  FPGA Fabric & Artix-7 \\
  DDR3 RAM & 512 MB \\
  Flash Storage & 32 MB (quad SPI) \\
\end{DndTable}

\subsection{Physical Characteristics}

\begin{DndTable}[header=Physical Specifications]{lX}
  \textbf{Parameter} & \textbf{Value} \\
  Dimensions & 69mm × 30mm × 8mm \\
  Weight & Approximately 10 grams \\
  Power Consumption & 2W typical, 5V via USB \\
  Operating Temp & 0°C to 70°C \\
  RF Connectors & 2× SMA female (TX and RX) \\
  Host Interface & USB 2.0 OTG (480 Mbps) \\
\end{DndTable}

\section{RF Performance Characteristics}

\subsection{Receiver Performance}

\begin{DndTable}[header=RX Specifications]{lX}
  \textbf{Parameter} & \textbf{Typical Value} \\
  Noise Figure & <4 dB @ max gain \\
  Gain Range & 0 to 76 dB (1 dB steps) \\
  Input Power Range & -90 dBm to 0 dBm \\
  DC Offset & <10 LSB \\
  RX Input Impedance & 50$\Omega$ \\
\end{DndTable}

\subsection{Transmitter Performance}

\begin{DndTable}[header=TX Specifications]{lX}
  \textbf{Parameter} & \textbf{Typical Value} \\
  Output Power & -89 dB to 0 dB (adjustable) \\
  TX Output Impedance & 50$\Omega$ \\
  EVM & <3\% @ max power \\
\end{DndTable}

\begin{DndSidebar}[float=!t]{Power Output Warning}
  The PlutoSDR's maximum output power is approximately 7 dBm into 50$\Omega$. For applications requiring higher power, use an external RF power amplifier. Always ensure proper attenuation when connecting TX output directly to sensitive receivers to avoid damage. Although digital attenuation ranges to 0 dB, the on-board RF chain typically delivers +5 to +7 dBm maximum output power into 50 $\Omega$
\end{DndSidebar}

%
%
% Write a section on modifying to 2TX/2RX
% Write a section on modifying to 2TX/2RX 

%
%
\section{MATLAB Setup (Windows / MacOS)}

This section describes the installation and configuration procedure for using the PlutoSDR with MATLAB on Windows and macOS systems. The workflow is identical across platforms, with minor differences in driver installation and device recognition.

\subsection{Required MATLAB Components}

To interface with the PlutoSDR, the following MATLAB components are required:

\begin{DndTable}[header=Required Software]{lX}
  \textbf{Component} & \textbf{Notes} \\
  MATLAB & R2019a or later \\
  Communications Toolbox & Mandatory for SDR support \\
  PlutoSDR Support Package & Vendor-specific hardware support \\
  USB Drivers & Platform-specific (automatic on macOS) \\
\end{DndTable}

Earlier MATLAB versions may function but can exhibit instability at higher sample rates or during long continuous streaming sessions.

\subsection{Installing the PlutoSDR Support Package}

The PlutoSDR interface is provided through the \emph{Communications Toolbox Support Package for Analog Devices ADALM-PLUTO Radio}.

\begin{lstlisting}[language=Matlab]
% Launch support package installer
supportPackageInstaller
\end{lstlisting}

In the installer:
\begin{enumerate}
  \item Select \textbf{Communications Toolbox Support Package for Analog Devices ADALM-PLUTO Radio}
  \item Follow the on-screen installation steps
  \item Restart MATLAB after installation completes
\end{enumerate}

MATLAB automatically installs all required backend interface binaries during this process.

\subsection{Windows-Specific Setup}

On Windows systems, the PlutoSDR is seen as a composite USB device and requires a kernel-mode driver.

\begin{DndComment}{Windows Driver Installation}
  MATLAB installs the required driver automatically as part of the support package. If the device is not recognized, verify that the PlutoSDR appears under \texttt{libusb-based USB devices} or \texttt{USB Devices} in Device Manager.
\end{DndComment}

\textbf{Common Windows Issues:}
\begin{itemize}
  \item USB driver conflicts with older SDR software
  \item Insufficient USB controller bandwidth (use a native USB port, not a hub)
  \item Power management suspending the USB device
\end{itemize}

Disabling USB selective suspend in Windows Power Options is recommended for long streaming sessions.

\subsection{macOS-Specific Setup}

macOS systems do not require manual driver installation. The PlutoSDR uses a user-space USB driver provided by MATLAB.

\begin{DndComment}{macOS Permissions}
  On first use, macOS may request permission for MATLAB to access removable devices or network interfaces. These permissions must be granted for proper operation.
\end{DndComment}

\textbf{Apple Silicon (M-series) Notes:}
\begin{itemize}
  \item MATLAB must run natively (Arm64) or under Rosetta consistently
  \item USB throughput is generally reliable but sensitive to background load
  \item High-sample-rate continuous streaming benefits from closing other USB-heavy applications
\end{itemize}

\subsection{Verifying PlutoSDR Detection}

After installation, verify that MATLAB detects the PlutoSDR:

\begin{lstlisting}[language=Matlab]
>> sdrinfo('Pluto')
\end{lstlisting}

A valid response lists device serial number, firmware version, and connection status. Absence of output indicates a driver or connection issue.

\subsection{Firmware Version Compatibility}

MATLAB expects a PlutoSDR firmware version compatible with the installed support package.

\begin{DndTable}[header=Firmware Considerations]{lX}
  \textbf{Aspect} & \textbf{Recommendation} \\
  Factory Firmware & Works with most MATLAB versions \\
  Custom Firmware & Must retain IIO and MATLAB compatibility \\
  Frequency Extensions & Supported by MATLAB but not guaranteed \\
\end{DndTable}

Firmware updates can be performed via the PlutoSDR web interface at \texttt{http://192.168.2.1}. After updating firmware, power-cycle the device before reconnecting to MATLAB.

\subsection{Basic Connection Test}

A minimal end-to-end functional test:

\begin{lstlisting}[language=Matlab]
% Create transmit object
tx = sdrtx('Pluto');
tx.CenterFrequency = 1e9;              
tx.BasebandSampleRate = 1e6;           
tx.Gain = -20;                         

% Generate a complex baseband tone
Ns = 1024;
fTone = 50e3;                         
t = (0:Ns-1).' / tx.BasebandSampleRate;

txData = exp(1j * 2 * pi * fTone * t);

% Create receiver object
rx = sdrrx('Pluto');
rx.CenterFrequency = 1e9;             
rx.BasebandSampleRate = 1e6;          
rx.SamplesPerFrame = Ns;
rx.GainSource = 'Manual';
rx.Gain = 20;                       

% Transmit one frame
tx(txData);

% Receive samples
rxData = rx();

% Plot transmitted and received signals
figure;
subplot(2,1,1);
plot(real(txData));
title('Transmitted Signal (Real Part)');
xlabel('Sample Index');
ylabel('Amplitude');
grid on;

subplot(2,1,2);
plot(real(rxData));
title('Received Signal (Real Part)');
xlabel('Sample Index');
ylabel('Amplitude');
grid on;

% Release hardware
release(rx);
release(tx);
\end{lstlisting}

Successful execution without errors confirms proper installation, driver functionality, and USB communication.

\subsection{Recommended MATLAB Settings}

For reliable operation:

\begin{DndTable}[header=Recommended Settings]{lX}
  \textbf{Setting} & \textbf{Recommendation} \\
  Sample Rate & $\leq 4$ MSPS (full duplex) \\
  Data Type & \texttt{int16} for streaming \\
  USB Cable & Short, shielded USB 2.0 cable \\
  Power Source & Direct host USB port \\
\end{DndTable}

For extended recording or real-time applications, preallocate all memory buffers and avoid dynamic object reconfiguration inside processing loops.

\subsection{Troubleshooting Checklist}

\begin{itemize}
  \item Ensure PlutoSDR is not assigned to another application
  \item Verify IP connectivity at \texttt{192.168.2.1}
  \item Restart MATLAB after connecting the device
  \item Power-cycle the PlutoSDR if USB enumeration fails
\end{itemize}

\begin{DndReadAloud}
  Most MATLAB--PlutoSDR issues originate from USB bandwidth limitations or driver conflicts rather than RF configuration errors. Always validate connectivity with a minimal receiver test before attempting complex signal processing pipelines.
\end{DndReadAloud}
\section{PySDR setup on Linux}
\noindent
\textbf{Important Installation Instructions:}

\medskip
The installation of all required libraries, software, and firmware \textbf{must be performed strictly through the official documentation} available at:
\begin{center}
\url{https://pysdr.org/content/pluto.html#}
\end{center}

\noindent
It is \textbf{mandatory} to execute the commands provided on the official webpage \textbf{exactly in the specified order} and \textbf{one command at a time}. Skipping steps, altering the order, or executing multiple commands simultaneously may lead to installation failures or unstable system behavior.

\medskip
Furthermore, it is \textbf{strongly recommended} to perform the entire setup inside a \textbf{dedicated Python virtual environment} rather than using the global Python installation. This helps prevent dependency conflicts and ensures a clean, reproducible setup.

After completing the setup, it is \textbf{mandatory to verify the correctness of the installation} by executing the \textbf{test examples provided on the official PySDR webpage}. These tests must be performed for:
\begin{itemize}
    \item the \textbf{transmitter},
    \item the \textbf{receiver}, and
    \item the \textbf{simultaneous operation of both transmitter and receiver}.
\end{itemize}

\noindent
Successful execution of all the above tests confirms the proper functioning of the software, firmware, and hardware interfaces.

After this step, we also suggest giving a good read to the examples on synchronization provided at
\href{https://pysdr.org/content/sync.html}{pysdr.org}. These examples give the reader a very good
understanding of important topics in communication systems.
\section{Python Driver Installation and Configuration}

This section describes the installation and configuration of PlutoSDR drivers for Python across Windows, Linux, and macOS platforms. The recommended approach uses Miniconda-based virtual environments to ensure reproducibility and avoid dependency conflicts.

\subsection{Prerequisites}

Before proceeding, ensure familiarity with basic Python programming, digital signal processing concepts, and command-line usage. The PlutoSDR must be physically connected via USB and recognized by the operating system.

\subsection{Cross-Platform Installation Guidance}

General installation guidance applicable to all platforms is maintained by Analog Devices:

\begin{lstlisting}
https://wiki.analog.com/sdrseminars
\end{lstlisting}

This resource contains platform-specific driver requirements, troubleshooting steps, and supported software versions.

\subsection{Recommended Installation for Windows}

The following procedure provides a reliable installation path for Windows users using Miniconda for environment management.

\subsubsection{Install Miniconda}

Download and install Miniconda from the official Anaconda distribution:

\begin{lstlisting}
https://www.anaconda.com/docs/getting-started/miniconda/install
\end{lstlisting}

Miniconda provides a minimal Python distribution with the \texttt{conda} package manager, suitable for isolated environments.

\subsubsection{Create and Activate a Virtual Environment}

Open Command Prompt or PowerShell and execute:

\begin{lstlisting}[language=bash]
conda create -n pluto python=3.10
conda activate pluto
\end{lstlisting}

Python 3.10 is recommended for compatibility with current \texttt{pyadi-iio} releases.

\subsubsection{Install PyADI-IIO}

With the environment active, install the Analog Devices Python interface:

\begin{lstlisting}[language=bash]
pip install pyadi-iio
\end{lstlisting}

This installs the \texttt{adi} module used to interface with the PlutoSDR.

\subsubsection{Verify Installation}

Create a test script \texttt{pluto\_test.py}:

\begin{lstlisting}[language=Python]
import numpy as np
import adi

sample_rate = 1e6
center_freq = 915e6

sdr = adi.Pluto("ip:192.168.2.1")
sdr.sample_rate = int(sample_rate)
sdr.tx_rf_bandwidth = int(sample_rate)
sdr.tx_lo = int(center_freq)
sdr.tx_hardwaregain_chan0 = -50

N = 10000
t = np.arange(N) / sample_rate
samples = 0.5 * np.exp(2j * np.pi * 100e3 * t)
samples *= 2**14

for _ in range(100):
    sdr.tx(samples)
\end{lstlisting}

Run:

\begin{lstlisting}[language=bash]
python pluto_test.py
\end{lstlisting}

Successful execution indicates correct installation and connectivity.

\begin{DndComment}{Common Issues}
If you encounter \texttt{ModuleNotFoundError: No module named 'adi'}, verify that the Conda environment is active and \texttt{pyadi-iio} is installed within it.
\end{DndComment}

\subsubsection{Extended TX/RX Test}

Example TX/RX scripts are available at:

\begin{lstlisting}
https://pysdr.org/content/pluto.html
\end{lstlisting}

RX and TX hardware gains may require manual tuning for clean visualization.

\section{Frequency Range Extension}

By default, the PlutoSDR supports approximately 325~MHz to 3.8~GHz. The underlying RFIC is capable of operation from 70~MHz to 6~GHz, which can be unlocked through firmware configuration.

\subsection{Performance Considerations}

Operation outside the official frequency range is not guaranteed. Performance degradation, increased phase noise, and instability may occur, particularly above 5~GHz.

\subsection{Configuration Procedure}

\subsubsection{Verify Firmware Version}

Firmware version 0.31 or newer is required:

\begin{lstlisting}
https://wiki.analog.com/university/tools/pluto/users/firmware
\end{lstlisting}

\subsubsection{Enable Extended Tuning}

SSH into the device:

\begin{lstlisting}[language=bash]
ssh root@192.168.2.1
\end{lstlisting}

Run:

\begin{lstlisting}[language=bash]
fw_setenv attr_name compatible
fw_setenv attr_val ad9364
reboot
\end{lstlisting}

After reboot, verify tuning outside the default range.

\section{Dual Transmit and Receive Configuration (2TX/2RX)}

The AD9361 supports two TX and two RX chains internally, though only one of each is routed externally on stock hardware.

\subsection{Hardware Modification Requirements}

Full 2TX/2RX operation requires hardware modification:

\begin{lstlisting}
https://wiki.analog.com/university/tools/pluto/hacking/hardware
\end{lstlisting}

\begin{DndReadAloud}
Hardware modification voids warranty and carries risk. Proceed only if you have appropriate soldering experience.
\end{DndReadAloud}

\subsection{Firmware Configuration}

Enable 2TX/2RX mode:

\begin{lstlisting}[language=bash]
ssh root@192.168.2.1
fw_setenv attr_name compatible
fw_setenv attr_val ad9361
fw_setenv mode 2r2t
reboot
\end{lstlisting}

\subsection{Python Configuration Example}

\begin{lstlisting}[language=Python]
import numpy as np
import adi

sdr = adi.ad9361("ip:192.168.2.1")
sdr.sample_rate = int(2e6)

sdr.tx_lo = int(2.4e9)
sdr.tx_enabled_channels = [0, 1]
sdr.tx_hardwaregain_chan0 = -30
sdr.tx_hardwaregain_chan1 = -30

sdr.rx_lo = int(2.4e9)
sdr.rx_enabled_channels = [0, 1]
sdr.rx_hardwaregain_chan0 = 20
sdr.rx_hardwaregain_chan1 = 20

N = 1024
t = np.arange(N) / sdr.sample_rate
tx0 = np.exp(2j * np.pi * 100e3 * t) * 2**14
tx1 = np.exp(2j * np.pi * 200e3 * t) * 2**14

sdr.tx([tx0, tx1])
rx_data = sdr.rx()
\end{lstlisting}

\subsection{MATLAB Configuration Summary}

MATLAB requires the Analog Devices Transceiver Toolbox and the AD9361 system object:

\begin{lstlisting}[language=Matlab]
tx = adi.AD9361.Tx('uri','ip:192.168.2.1');
tx.EnabledChannels = [1,2];

rx = adi.AD9361.Rx('uri','ip:192.168.2.1');
rx.EnabledChannels = [1,2];
\end{lstlisting}

\subsection{Reference Implementations}

\begin{DndTable}[header=Selected Resources]{lX}
\textbf{Topic} & \textbf{Resource} \\
2TX/2RX Setup & \href{https://www.youtube.com/watch?v=ph0Kv4SgSuI}{YouTube} \\
Phased Arrays & \href{https://github.com/jonkraft}{Jon Kraft GitHub} \\
ADI Python Examples & \href{https://github.com/analogdevicesinc/pyadi-iio/tree/master/examples}{GitHub} \\
\end{DndTable}


\section{System Limitations}

\subsection{USB Interface Bandwidth}

The PlutoSDR uses USB 2.0 (480 Mbps theoretical), which imposes practical limitations on sustained data rates.

\begin{DndComment}{USB Bandwidth Reality}
  While USB 2.0 provides 480 Mbps theoretical bandwidth, real-world performance is typically limited to 30-35 MB/s (240-280 Mbps) due to protocol overhead. For IQ data (16-bit I + 16-bit Q = 4 bytes per sample), this limits sustained sample rates to approximately 4-5 MSPS in each direction simultaneously.
\end{DndComment}

\begin{DndTable}[header=Practical Data Rate Limits]{lX}
  \textbf{Configuration} & \textbf{Maximum Sustained Rate} \\
  RX Only (1T0R) & ~8 MSPS \\
  TX Only (0T1R) & ~8 MSPS \\
  Full Duplex (1T1R) & ~4 MSPS per path while switching TX buffers\\
\end{DndTable}

\subsection{Sample Rate Considerations}

\begin{DndTable}[header=Sample Rate Guidelines]{lX}
  \textbf{Rate Range} & \textbf{Characteristics} \\
  2.084 - 4 MSPS & Full duplex, no USB bottleneck \\
  4 - 8 MSPS & May drop samples in full duplex \\
  8 - 20 MSPS & RX or TX only, buffering required \\
  20 - 61.44 MSPS & Intermittent operation, not sustained \\
\end{DndTable}



\begin{DndReadAloud}
  For reliable continuous operation without dropped samples or buffer overruns, limit your sample rate to 4 MSPS or less when operating in full-duplex mode. Higher rates can be achieved in half-duplex or with careful buffer management and may require reducing processing overhead on the host computer.
\end{DndReadAloud}

\subsection{Frequency and Bandwidth Limits}

\begin{DndTable}[header=RF Configuration Limits]{lXX}
  \textbf{Parameter} & \textbf{Minimum} & \textbf{Maximum} \\
  Center Frequency & 70 MHz* & 6000 MHz* \\
  RF Bandwidth & 200 kHz & 56 MHz \\
  Sample Rate & 2.083333 MSPS & 61.44 MSPS \\
\end{DndTable}

\begin{DndSidebar}[float=!b]{Extended Frequency Range}
  *The officially supported frequency range is 325 MHz to 3.8 GHz. Operation from 70 MHz to 6 GHz is possible but not guaranteed. Performance degrades outside the official range, particularly below 300 MHz and above 4 GHz. Always verify performance for your specific application and frequency.
\end{DndSidebar}

\subsection{Memory and Processing Constraints}

\begin{DndTable}[header=System Resources]{lX}
  \textbf{Resource} & \textbf{Limitation} \\
  DDR3 RAM & 512 MB (shared with Linux OS) \\
  Available RAM & ~200-300 MB for buffers \\
  Flash Storage & 32 MB (firmware and config) \\
  ARM Processing & Limited DSP capability \\
  FPGA Resources & Pre-configured, not user-accessible \\
\end{DndTable}

The PlutoSDR is designed primarily as an RF front-end, not a standalone signal processing platform. Heavy DSP operations should be performed on the host computer, not on the Zynq processor.

\section{Network and Connectivity}

\subsection{Connection Modes}

\begin{DndTable}[header=Connectivity Options]{lX}
  \textbf{Mode} & \textbf{Description} \\
  USB Device & Default mode, PlutoSDR as USB peripheral \\
  USB Host & PlutoSDR acts as host (requires OTG adapter) \\
  Ethernet & USB-Ethernet bridge (192.168.2.1) \\
  Mass Storage & Configuration file access \\
  Serial Console & UART access for debugging \\
\end{DndTable}

\subsection{IP Configuration}

When connected via USB, the PlutoSDR creates a virtual Ethernet interface:

\begin{lstlisting}[language=bash]
# Default PlutoSDR IP address
192.168.2.1

# Default host computer IP (assigned automatically)
192.168.3.1

# Access web interface
http://192.168.2.1

# Screen (tested / works)
sudo screen /dev/tty__ 115200

# SSH access (also works)
ssh root@192.168.2.1
# Default password: analog
\end{lstlisting}

\section{Regulatory and Safety}
We only put this because we are obligated to, legally.  Really, nothing to be concerned about unless you are actively trying to do damage. (But it is useless by itself, for almost all illegal activities other than GPS spoofing, but you signed a contract, which holds you liable so, upto you)
\begin{DndReadAloud}
  The ADALM-PLUTO is an educational development tool intended for laboratory and learning environments. It is NOT certified for commercial transmission in any regulated spectrum. Users are responsible for ensuring compliance with local regulations when transmitting RF signals. In many jurisdictions, transmitting on licensed frequencies without authorization is illegal.
\end{DndReadAloud}

\section{Quick Reference Summary}

\begin{DndTable}[header=PlutoSDR at a Glance, color=PhbLightGreen]{lX}
  \textbf{Specification} & \textbf{Value} \\
  \hline
  RF IC & AD9361 Agile Transceiver \\
  Frequency Range & 325 MHz - 3.8 GHz (70 MHz - 6 GHz extended) \\
  Channels & 1T1R (expandable to 2T2R) \\
  Sample Rate & 2.083 - 61.44 MSPS \\
  Bandwidth & 200 kHz - 56 MHz \\
  Interface & USB 2.0 (480 Mbps) \\
  Sustained Rate & ~4 MSPS full duplex \\
  Processor & Zynq Z-7010 (Dual ARM + FPGA) \\
  Memory & 512 MB DDR3 \\
  Connectors & 2× SMA (RX/TX) \\
  Power & 5V USB, 2W typical \\
\end{DndTable}