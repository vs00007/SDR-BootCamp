% ============================================================================
% PLUTO SDR MANUAL - STYLE GUIDE AND COMMAND REFERENCE
% ============================================================================
% Keep this file handy while writing chapters!
% Save as: style_guide.tex (for reference only, not compiled)
% ============================================================================

% ============================================================================
% CHAPTER FILE STRUCTURE
% ============================================================================
% Each chapter file should NOT include document preamble or \begin{document}
% Start directly with chapter content:

\chapter{Your Chapter Title}

\DndDropCapLine{F}{irst sentence of chapter} continues here with the rest
of the introductory paragraph.

% ============================================================================
% SECTION COMMANDS
% ============================================================================

\section{Main Section}
% Use for major topics within a chapter

\subsection{Subsection}
% Use for sub-topics within a section

\subsubsection{Subsubsection}
% Use for further divisions

\paragraph{Paragraph Header}
% Inline header, rarely used

\subparagraph{Subparagraph Header}
% Smaller inline header


% ============================================================================
% SPECIAL SECTION HEADERS
% ============================================================================

% For Feature/Capability Headers (like D&D feats):
\DndFeatHeader{Feature Name}[Prerequisites or conditions]
Description of the feature or capability.

% Example:
\DndFeatHeader{Buffer Optimization}[Requires: libiio 0.21+]
When using kernel buffers larger than 256k samples, you gain advantage
on throughput checks and reduce latency by 50\%.


% For Item/Tool Headers (like magic items):
\DndItemHeader{Tool Name}{Category, rarity/type}
Description of the tool or component.

% Example:
\DndItemHeader{iio\_readdev}{Command-line tool, essential}
This tool has unlimited uses per day. While using it, you can capture
samples from any IIO device and redirect them to a file or pipe.


% For Procedure/Function Headers (like spells):
\DndSpellHeader%
  {Function Name}
  {Category or return type}
  {Parameters}
  {Scope}
  {Dependencies}
  {Constraints}
Description of the function or procedure.

% Example:
\DndSpellHeader%
  {iio\_buffer\_push()}
  {ssize\_t return type}
  {struct iio\_buffer *buf}
  {User-space to kernel-space}
  {libiio.h}
  {Blocking until buffer space available}
Transfers samples from user-space buffer to the kernel, making them
available for transmission via the DAC.


% ============================================================================
% TEXT BOXES AND CALLOUTS
% ============================================================================

% READ ALOUD BOX - For important instructions or step-by-step procedures
\begin{DndReadAloud}
  To initialize your first buffer, you must first create an IIO context.
  Connect to your Pluto SDR via USB and verify the device is enumerated
  properly using lsusb. Only then can you proceed to buffer creation.
\end{DndReadAloud}


% COMMENT BOX - For tips, notes, and additional information
% This is breakable across columns
\begin{DndComment}{Tip: Buffer Sizes}
  A |DndComment| is perfect for tips and notes. Always choose buffer
  sizes that are powers of 2 for optimal DMA performance. Start with
  4096 samples and adjust based on your application's latency requirements.
\end{DndComment}


% SIDEBAR - For related information, warnings, or deep dives
% This is NOT breakable, best used with float positioning
\begin{DndSidebar}[float=!t]{Understanding Sample Rates}
  The AD9361 can operate at sample rates from 2.083 MSPS to 61.44 MSPS.
  The actual achievable rate depends on your host interface (USB 2.0
  limits you to approximately 4 MSPS sustained). Consider using Ethernet
  for higher throughput applications.
\end{DndSidebar}

% Float options: !t (top), !b (bottom), !h (here)


% ============================================================================
% TABLES
% ============================================================================

% Basic table with automatic row coloring
\begin{DndTable}[header=Table Title]{XX}
  \textbf{Column 1} & \textbf{Column 2} \\
  Data 1 & Data 2 \\
  Data 3 & Data 4 \\
  Data 5 & Data 6 \\
\end{DndTable}

% Column specifiers:
% X = flexible width column (fills available space)
% l = left-aligned fixed width
% c = center-aligned fixed width
% r = right-aligned fixed width
% p{width} = paragraph column with specified width

% Table with custom color
\begin{DndTable}[header=IIO Attributes, color=PhbLightGreen]{lXc}
  \textbf{Attribute} & \textbf{Description} & \textbf{R/W} \\
  sampling\_frequency & Current sample rate in Hz & R/W \\
  rf\_bandwidth & RF filter bandwidth in Hz & R/W \\
  gain\_control\_mode & AGC mode: manual/fast/slow & R/W \\
\end{DndTable}

% Full-width table (spans both columns)
\begin{table*}[t]
  \begin{DndTable}[width=\linewidth, header=Buffer Functions]{lXX}
    \textbf{Function} & \textbf{Purpose} & \textbf{Blocking} \\
    iio\_buffer\_create() & Allocate buffer & No \\
    iio\_buffer\_refill() & Receive samples & Yes \\
    iio\_buffer\_push() & Transmit samples & Yes \\
  \end{DndTable}
\end{table*}


% ============================================================================
% CODE LISTINGS
% ============================================================================

% Inline code with vertical bars (already configured with \MakeShortVerb)
Use the |iio_context_create()| function to initialize.

% Code blocks - C language (default in main file)
\begin{lstlisting}
struct iio_context *ctx;
struct iio_device *dev;
struct iio_buffer *buf;

ctx = iio_context_create_default();
dev = iio_context_find_device(ctx, "cf-ad9361-lpc");
buf = iio_device_create_buffer(dev, 4096, false);
\end{lstlisting}

% Code blocks - Python
\begin{lstlisting}[language=Python]
import iio

ctx = iio.Context()
dev = ctx.find_device("cf-ad9361-lpc")
buf = iio.Buffer(dev, 4096)
\end{lstlisting}

% Code blocks - Bash/Shell
\begin{lstlisting}[language=bash]
iio_info
iio_attr -a
iio_readdev -s 4096 cf-ad9361-lpc > samples.dat
\end{lstlisting}


% ============================================================================
% MAP REGIONS (for hierarchical structures)
% ============================================================================

% Use for documenting system architecture or signal flow
\DndArea{IIO Device Hierarchy}
The top-level structure containing all IIO devices.

\DndSubArea{cf-ad9361-lpc (RX Device)}
The receive path device with voltage channels.

\DndSubArea{cf-ad9361-dds-core-lpc (TX Device)}
The transmit path device with voltage output channels.

\DndArea{Buffer Memory Regions}
Memory allocation for sample buffers.

\DndSubArea{Kernel Space DMA Buffer}
Direct memory access region managed by the kernel driver.

\DndSubArea{User Space Buffer}
Application-accessible memory mapped from kernel space.


% ============================================================================
% MONSTER/COMPONENT STAT BLOCKS
% ============================================================================

% Use for detailed component or function specifications
\begin{DndMonster}{AD9361 Transceiver}
  \DndMonsterType{RF Transceiver IC (Analog Devices), highly configurable}
  
  \DndMonsterBasics[
    armor-class = {Frequency Range: 70 MHz to 6 GHz},
    hit-points  = {Bandwidth: 200 kHz to 56 MHz},
    speed       = {Sample Rate: 2.083 to 61.44 MSPS},
  ]

  \DndMonsterAbilityScores[
    str = 12, % TX Power
    dex = 16, % Frequency Agility
    con = 14, % Noise Figure
    int = 18, % Programmability
    wis = 15, % Filtering
    cha = 13, % Spectrum Quality
  ]

  \DndMonsterDetails[
    senses = {12-bit ADC, 12-bit DAC resolution},
    languages = {SPI control interface, IIO subsystem},
    challenge = Advanced,
  ]

  \DndMonsterAction{Receive Mode}
  Configure the transceiver for signal reception with programmable
  gain and filtering.

  \DndMonsterAction{Transmit Mode}
  Generate and transmit RF signals with adjustable attenuation.

  \DndMonsterSection{Key Attributes}
  \begin{itemize}
    \item Integrated frequency synthesizers
    \item Automatic and manual gain control
    \item Programmable FIR filters
    \item Fast frequency hopping capability
  \end{itemize}
\end{DndMonster}


% ============================================================================
% COLORS
% ============================================================================

% Available theme colors (from core D&D books):
% PhbLightGreen  - Light green (default, Part 1 style)
% PhbLightCyan   - Light cyan (Part 2 style) - RECOMMENDED FOR SDR MANUAL
% PhbMauve       - Pale purple (Part 3 style)
% PhbTan         - Light brown (Appendix style)
% DmgLavender    - Pale purple
% DmgCoral       - Orange-pink
% DmgSlateGray   - Blue-gray
% DmgLilac       - Purple-gray

% Change theme for a section:
\begingroup
\DndSetThemeColor[PhbMauve]
% Your colored content here
\endgroup


% ============================================================================
% TYPOGRAPHY
% ============================================================================

% Drop cap at chapter start (already shown above)
\DndDropCapLine{F}{irst letter is large} and rest continues normally.

% Emphasis
\emph{emphasized text}

% Strong emphasis
\textbf{bold text}

% Typewriter (for code, filenames, commands)
\texttt{filename.txt}

% Small caps
\textsc{Small Caps}


% ============================================================================
% CROSS-REFERENCES
% ============================================================================

% Label a section
\section{Buffer Management}\label{sec:buffer_mgmt}

% Reference it
See Section~\ref{sec:buffer_mgmt} for details.

% Label a table
\begin{DndTable}[header=My Table]{XX}\label{tab:mytable}
  % table content
\end{DndTable}

% Reference it
As shown in Table~\ref{tab:mytable}, the buffer sizes...


% ============================================================================
% LISTS
% ============================================================================

% Bullet list
\begin{itemize}
  \item First item
  \item Second item
  \item Third item
\end{itemize}

% Numbered list
\begin{enumerate}
  \item First step
  \item Second step
  \item Third step
\end{enumerate}

% Description list
\begin{description}
  \item[Term 1] Definition of term 1
  \item[Term 2] Definition of term 2
\end{description}


% ============================================================================
% FIGURES AND IMAGES
% ============================================================================

% Single column figure
\begin{figure}[h]
  \centering
  \includegraphics[width=0.9\linewidth]{images/buffer_diagram.png}
  \caption{IIO Buffer Architecture}
  \label{fig:buffer_arch}
\end{figure}

% Full-width figure (spans both columns)
\begin{figure*}[t]
  \centering
  \includegraphics[width=0.9\textwidth]{images/signal_flow.png}
  \caption{Complete Signal Flow Diagram}
  \label{fig:signal_flow}
\end{figure*}


% ============================================================================
% MATHEMATICAL EXPRESSIONS
% ============================================================================

% Inline math
The Nyquist rate is $f_s \geq 2f_{max}$.

% Display math
\[
  SNR = 20 \log_{10}\left(\frac{V_{signal}}{V_{noise}}\right)
\]


% ============================================================================
% BEST PRACTICES FOR COLLABORATION
% ============================================================================

% 1. Use consistent indentation (2 or 4 spaces)
% 2. Add comments before complex tables or code blocks
% 3. Keep lines under 80 characters when possible
% 4. Use labels for all sections, tables, and figures you might reference
% 5. Test compile frequently to catch errors early
% 6. Use meaningful label names: sec:buffer_mgmt not sec:1
% 7. Keep one sentence per line for easier diff/merge
% 8. Use \include{} not \input{} for chapter files


% ============================================================================
% EXAMPLE CHAPTER SNIPPET
% ============================================================================

\chapter{Understanding Buffers}

\DndDropCapLine{B}{uffers are the fundamental mechanism} for transferring
sample data between your application and the Pluto SDR hardware.
Understanding how buffers work in the IIO framework is essential for
achieving optimal performance.

\section{Buffer Architecture}

The IIO subsystem uses a circular buffer architecture that operates
in kernel space. Your application interacts with these buffers through
the libiio library.

\begin{DndComment}{Memory Efficiency}
  Circular buffers eliminate the need for frequent memory allocations.
  The kernel continuously fills (RX) or drains (TX) the buffer while
  your application processes data at its own pace.
\end{DndComment}

\subsection{Kernel Space vs User Space}

\begin{DndTable}[header=Buffer Regions]{lX}
  \textbf{Region} & \textbf{Description} \\
  Kernel Space & DMA-accessible memory managed by driver \\
  User Space & Memory-mapped region accessible to application \\
  Transfer Zone & Synchronization point between spaces \\
\end{DndTable}

\section{Creating Your First Buffer}

\begin{DndReadAloud}
  To create a buffer, you must first have a valid IIO device handle.
  The buffer size must be specified in samples, not bytes. Choose a
  size appropriate for your latency requirements.
\end{DndReadAloud}

\begin{lstlisting}
// Create a 4096-sample buffer for RX
struct iio_buffer *rxbuf;
rxbuf = iio_device_create_buffer(rx_dev, 4096, false);
if (!rxbuf) {
    perror("Failed to create buffer");
    return -1;
}
\end{lstlisting}

The |false| parameter indicates this is a receive buffer (input).
For transmit buffers, use |true|.


% ============================================================================
% END OF STYLE GUIDE
% ============================================================================